\documentclass[]{article}
\usepackage{kotex}
\usepackage{amsmath}

%opening
\title{Systematic Unvertainty Note}
\author{}

\begin{document}

\maketitle

중력가속도 실험에 대한 Systematic Uncertainty Estimation

Based on Monte-Carlo Simulation

\section{Equation of Datapoints}

\paragraph{Basic Equation}
\begin{equation}
h = \frac{1}{2} g t^2
\end{equation}
\begin{equation}
t = \sqrt{2 \frac{g}{h}}
\end{equation}

\paragraph{Basic Equation with uncertainty}

\begin{equation}
h+\delta h = \frac{1}{2} g \left(t+\delta t\right)^2
\end{equation}

\begin{equation}
t = \sqrt{2 \frac{g}{h+\delta h}} + \delta t
\end{equation}

with following random distribution

\begin{table}[!h]
	\begin{tabular}{c|cc}
		Random Variable & Distribution & Reason\\\hline
		$\delta h$ & Gaussian Disribution & Measured by human-eye\\
		$\delta t$ & Uniform Disribution & Measured by mechanical clock
	\end{tabular}	
\end{table}

With overlapped uncertainty (depending on measuring methods), the $\delta h$ has different value for each datapoints.


\section{Case Overlapped Length Dependencies}
If measuring length for each data-points has dependencies with each previous data-point. Uncertainty for each data-points is needed to declared especially.

\subsection{Error Propagation}

\paragraph{Basic Equation of uncertainty}
\begin{equation}
\delta f = \sqrt{\sum_i\left(\frac{\partial f}{\partial x_i} \delta x_i \right)^2}
\end{equation}

\paragraph{Overlapped Uncertainty, Propagated for Summation}

\begin{equation}
\delta (f+g) = \sqrt{(\delta g)^2 + (\delta f)^2  }
\end{equation}
For independent 2 formula $f$ and $g$.

Overlapping uncertainties. With measuring length period $l$, number $i$, single measuring uncertainty $\delta l_i$, Initial offset $b$ and its uncertainty $\delta b$

First term
\begin{align}
h_0 &= b \\
h_1 &= b + l + \delta l_1 \\
h_2 &= b + 2l +  \delta l_1 +\delta l_2
\end{align} 

with regularity, following correlation can be assumed.

\begin{equation}
h_n = b + n \cdot l + \sum_i{\delta l_i}
\end{equation}

Uncertainty is 
\begin{equation}
\delta h_i = \sqrt{\sum_i{\left(\delta l\right)^2}}
\end{equation}
$\delta l$ is constant for each $n$, 

\begin{equation}
\delta h_n = \sqrt{n} \delta l
\end{equation}



	

\end{document}
